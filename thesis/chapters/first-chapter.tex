\chapter{Panoramica su Vampire}
In questo capitolo viene presentato il theorem prover \textsc{Vampire}, la sua architettura e alcune
delle caratteristiche più importanti.

\section{Cos'è \textsc{Vampire}}
\textsc{Vampire} è un sistema che permette di provare la validità di teoremi (matematici e non) di logica del primo ordine. 
Il linguaggio di programmazione usato è C++ e la prima versione è stata sviluppata principalmente 
da Andrei Voronkov e Krystof Hoder tra il 1993 e il 1995 nell'Università di Manchester. Dopo il 1995, 
il progetto è stato sospeso e viene ripreso nel 1998. Dal '98 vengono implementate numerose versioni e, 
ancora oggi, proseguono gli sviluppi. \textsc{Vampire} ha vinto circa 45 titoli in diverse divisioni 
della CASC (CADE ATP System Competition) che è il campionato mondiale per gli ATP (Automated Theorem Prover).
Ecco alcune delle principali caratteristiche del sistema:
\begin{itemize}
    \item è molto veloce
    \item è portabile sulle piattaforme più comuni
    \item è semplice da usare
    \item ha strategie di ricerca con risorse limitate
    \item supporta numerose sintassi in input
    \item i vari tentativi per provare il problema possono essere parallelizzati su più processori
    \item può produrre, in base alle opzioni selezionate, output molto dettagliati 
\end{itemize}
\section{Architettura di \textsc{Vampire}}
\textsc{Vampire} ha un'architettura complessa ma, analizzandola da un alto livello, è possibile riconoscere tre moduli 
principali:
\begin{enumerate}
    \item \emph{parser}
    \item \emph{preprocessor}
    \item \emph{kernel}
\end{enumerate}
\begin{figure}[H]
    \incfig{architettura}
\end{figure}


\subsection{Parser}
Il \emph{parser} è un modulo adibito al parsing del problema. Il problema viene letto da un file e viene incapsulato in una classe.
Nel caso di un problema di logica del primo ordine con sintassi TPTP, allora viene diviso in più unità.
Ogni unità è una formula o una clausola a cui viene assegnato uno specifico tag in base al tipo (ipotesi, assioma, congettura, teorema, \dots).
Nel codice sorgente sono presenti le classi \verb|Problem|, \verb|Unit|, \verb|Formula|, \verb|Clause| e sono in relazione tra loro come mostrato nella figura \ref{fig:relazioni-classi}.
Oltre alle rappresentate \verb|NegatedFormula| e \verb|QuantifiedFormula|, sono presenti altre classi che ereditano \verb|Formula| come
\verb|AtomicFormula|, \verb|JunctionFormula|, \verb|BinaryFormula|, \dots
\begin{remark}
    Un' unità può essere una formula o una clausola ma non entrambe. La distinzione viene effettuata sulla base dell'
    attributo \emph{kind} che è un' enumerazione.
\end{remark}
\vspace{.3cm}
\begin{figure}[H]
    \begin{tikzpicture}
        \begin{class}{Problem}{-5,0}
            \attribute{- units : UnitList*}
        \end{class}
        \begin{class}{Unit}{-5,-2}
            \attribute{- kind : enum}
            \operation{+ getFormula()}
            \operation{+ asClause()}
        \end{class}
        \begin{class}{Formula}{-10,-5}
        \end{class}
        \begin{class}{Clause}{0,-5}
        \end{class}
        \begin{class}{QuantifiedFormula}{-10,-8}
            \inherit{Formula}
        \end{class}
        \begin{class}{NegatedFormula}{-4,-8}
            \inherit{Formula}
        \end{class}
    
        \unidirectionalAssociation{Problem}{}{}{Unit}
        \unidirectionalAssociation{Unit}{}{}{Formula}
        \unidirectionalAssociation{Unit}{}{}{Clause}
    \end{tikzpicture}
    \caption{Relazioni tra le classi}\label{fig:relazioni-classi}
\end{figure}
\subsection{Preprocessor}
Il \emph{\emph{preprocessor}} è un modulo che processa il problema in modo che sia trattabile dal \emph{kernel} in fase di risoluzione.
Per questo modulo è possibile abilitare numerose opzioni, inoltre vengono eseguite molte semplificazioni in modo da rendere il sistema il 
più veloce possibile. Di seguito sono descritti solo i passaggi fondamentali: 
\begin{description}
    \item[I step] \emph{Rectify} $\longrightarrow$ Il \emph{preprocessor} verifica se una formula ha variabili libere. Se la formula è aperta, 
    allora genera dei quantificatori per vincolare le variabili libere. Inoltre, verifica che per ogni variabile $x$ ci sia una sola occorrenza di $\exists x$ o $\forall x$. 
    \item[II step] \emph{Simplify} $\longrightarrow$ Il \emph{preprocessor} verifica se una formula contiene $\top$ o $\bot$. Nel caso la formula contenesse uno dei due, allora
    viene semplificata.
    \item[III step] \emph{Flatten} $\longrightarrow$ Il \emph{preprocessor} trasforma le formule in modo da renderle uniformi. Questo è solo uno step di formattazione che verrà eseguito più volte nel corso del preprocessing.
    \item[IV step] \emph{Unused definitions and pure predicate removal} $\longrightarrow$ Il \emph{preprocessor} rimuove i predicati e le definizioni delle funzioni che non vengono usati.
    \item[V step] \emph{ENNF} $\longrightarrow$ Il \emph{preprocessor} trasforma le formula in modo da ottenerle in \emph{extended negative normal form}.
    \begin{definition}
        Una formula è in \emph{extended negation normal form} se non contiene $\rightarrow$ e tutte le $\neg$ sono spostate, il più possibile, verso l'interno.  
    \end{definition}
    \item[VI step] \emph{Naming} $\longrightarrow$ Il \emph{preprocessor} definisce un nuovo predicato $p$ che viene usato come nome di una sotto-formula. 
    Questa tecnica viene utilizzata per evitare di generare un numero esponenziale di clausole nei prossimi passaggi di preprocessing.
    \begin{definition}
        Sia $\varphi|_\pi$ sotto-formula di $\varphi$ in posizione $\pi$ con variabili libere $\bar{x}$, allora si sostituisce $\varphi|_\pi$ con $p(\bar{x})$
        nuovo predicato e viene aggiunta la definizione $\text{def}(\varphi,\pi,p)$ tale che
        \[\text{def}(\varphi,\pi,p)=\begin{cases}
            \forall \bar{x} (p(\bar{x})\rightarrow \varphi|_\pi) & \text{se pol}(\varphi,\pi)=1\\
            \forall \bar{x} (\varphi|_\pi\rightarrow p(\bar{x})) & \text{se pol}(\varphi,\pi)=-1\\
            \forall \bar{x} (p(\bar{x})\leftrightarrow \varphi|_\pi) & \text{se pol}(\varphi,\pi)=0\\
        \end{cases}\]
        in cui $\text{pol}(\varphi,\pi)$ indica la polarità della sotto-formula di $\varphi$ in posizione $\pi$. $\text{pol}(\varphi,\pi)=0$ si verifica quando il simbolo di livello superiore di $\varphi|_\pi$ è 
        $\leftrightarrow$ o $\otimes$.   
    \end{definition}
    \begin{remark}
        Questa tecnica viene adottata solo se il numero di clausole generate supera una certa soglia. Infatti, in 
        \textsc{Vampire}, nella classe \verb|Naming|, è presente un attributo \emph{threshold} che indica proprio questo limite.
    \end{remark}
    \item[VII step] \emph{NNF} $\longrightarrow$ Il \emph{preprocessor} trasforma le formula in modo da ottenerle in \emph{negation normal form}.
    \begin{definition}
        Una formula è in \emph{negation normal form} se non contiene $\rightarrow,\leftrightarrow,\otimes$ e tutte le $\neg$ sono spostate, il più possibile, verso l'interno.
    \end{definition}
    \item[VIII step] \emph{Skolemization} $\longrightarrow$ Il \emph{preprocessor} applica la tecnica di skolemizzazione per eliminare i $\exists$ dalle formule.
    \begin{definition}
        Sia $F=\exists x\mid\varphi(y_1,\dots,y_n,x)$ formula in negation normal form allora la skolemizzazione si ottiene tramite la seguente sostituzione:
        \[F=\exists x\mid\varphi(y_1,\dots,y_n,x)\Rightarrow F'=\varphi(y_1,\dots,y_n,x)\{x \mapsto sk(y_1,\dots,y_n)\}\]
        in cui $\{x \mapsto sk(y_1,\dots,y_n)\}$ indica che $x$, la variabile precedentemente vincolata dal quantificatore esistenziale nella formula $F$, 
        è sostituita da una nuova funzione $sk(y_1,\dots,y_n)$, detta di Skolem, nella formula $F'$.
    \end{definition}
    \item[IX step] \emph{Clausification} $\longrightarrow$ Il \emph{preprocessor} applica la clausificazione su tutte le formule in modo da ottenere 
    un insieme di clausole.
    \begin{definition}
        La clausificazione è il risultato di:
        \begin{enumerate}
            \item $\forall x\mid\varphi(y_1,\dots,y_n,x)\Rightarrow\varphi(y_1,\dots,y_n,x)\{x \mapsto X\}$  in cui $X$ è una variabile designata che non occorre in $\varphi$
            \item Se una formula $F=\varphi'\land\varphi''$ allora viene spezzata in due unità diverse 
            $F'=\varphi'$ e $F''=\varphi''$
        \end{enumerate}
    \end{definition}  
\end{description} 
Alla fine di questi step, l'insieme di clausole risultanti è pronto per essere passato all'algoritmo di \emph{resolution}.
\subsection{Kernel}
Il \emph{kernel} è il sotto-sistema adibito alla risoluzione del problema. Per raggiungere questo obiettivo, 
viene implementato un algoritmo di saturazione che permette di trovare una confutazione all'insieme di clausole.
Questo è possibile tramite la saturazione dell'insieme con tutte le possibili inferenze presenti nel calcolo. 
\textsc{Vampire} possiede numerose inferenze ma è possibile sceglierne un sottoinsieme e definire un sistema 
d'inferenze per la logica del primo ordine.

Formalmente, per definire un sistema d'inferenze bisogna prima definire un \emph{simplification ordering} e una funzione di selezione.
\begin{definition}
    Un ordinamento $\succ$ sui termini è detto \emph{simplification ordering} se rispetta le seguenti condizioni:
    \begin{enumerate}
        \item $\succ$ è ben formato ovvero $\nexists t_0,t_1, \dots \text{sequenza infinita tale che } t_0 \succ t_1 \succ \dots$
        \item $\succ$ è monotona ovvero $l \succ r \rightarrow s(l)\succ s(r)$ per tutti i termini $s,l,r$
        \item $\succ$ è stabile per la sostituzione ovvero $l \succ r \rightarrow l\theta \succ r\theta$
        \item $\succ$ ha la \emph{subterm property} ovvero se $r$ sottotermine di $l$ e $l\neq r$ allora $l\succ r$   
    \end{enumerate}
\end{definition}
Questa definizione è estendibile agli atomi, ai letterali e alle clausole.
\begin{definition}
    Una funzione di selezione seleziona un sottoinsieme non vuoto di letterali in ogni clausola non vuota. In $\underline{L}\lor R$ \underline{L} indica il letterale selezionato.
\end{definition}
Il sistema di inferenze che viene definito di seguito è detto \emph{superposition inference system}.
\begin{definition}
    Il \emph{superposition inference system} è un sistema di inferenze composto dalle seguenti regole
    \begin{itemize}
        \item \textbf{Resolution}
        \begin{equation}
            \begin{gathered}
                \underline{\underline{A} \lor C_1 \quad\underline{\lnot A'}\lor C_2}\\
                (C_1 \lor C_2)\theta
            \end{gathered}
        \end{equation}
        in cui $\theta$ è l'unificatore più generale di $A$ e $A'$.
        \item \textbf{Factoring}
        \begin{equation}
            \begin{gathered}
                \underline{\underline{A} \lor \underline{A'} \lor C}\\
                (A \lor C)\theta
            \end{gathered}
        \end{equation}
        in cui $\theta$ è l'unificatore più generale di $A$ e $A'$.
        \item \textbf{Superposition}
        \begin{equation}
            \begin{gathered}
                \underline{\underline{l=r} \lor C_1 \quad \underline{L[s]}\lor C_2} \quad \underline{\underline{l=r} \lor C_1 \quad \underline{t[s]=t'}\lor C_2} \quad \underline{\underline{l=r} \lor C_1 \quad \underline{t[s]\neq t'}\lor C_2}\\
                (L[r] \lor C_1 \lor C_2)\theta \qquad (t[r]=t' \lor C_1 \lor C_2)\theta \qquad (t[r]\neq t' \lor C_1 \lor C_2)\theta
            \end{gathered}
        \end{equation}
        in cui $\theta$ è l'unificatore più generale di $l$ e $s$, $s$ non è una variabile, 
        $r\theta\prec l\theta$, solo nella prima regola $L[s]$ non è un \emph{equality literal}\footnote{Un \emph{equality literal} è una clausola 
        con un singolo letterale in cui è presente un =},
        \item \textbf{Equality resolution}
        \begin{equation}
            \begin{gathered}
                \underline{\underline{s\neq t} \lor C}\\
                C\theta
            \end{gathered}
        \end{equation}
        in cui $\theta$ è l'unificatore più generale di $s$ e $t$.
        \item \textbf{Equality factoring}
        \begin{equation}
            \begin{gathered}
                \underline{\underline{s=t} \lor \underline{s'=t'}\lor C}\\
                (s=t \lor t\neq t' \lor C)\theta
            \end{gathered}
        \end{equation}
        in cui $\theta$ è l'unificatore più generale di $s$ e $t$, $t\theta \prec s\theta$ e $t'\theta\prec t\theta$.
    \end{itemize}
\end{definition}
Se la funzione di selezione seleziona sia letterali negativi sia tutti i letterali massimali, allora è possibile enunciare la seguente proprietà: 
\begin{property}\label{sup}
    Il superposition inference system è sound e completo.
\end{property}
Una volta definito un sistema di inferenze, è possibile formalizzare il concetto di saturazione espresso precedentemente.
\begin{definition}
    Un insieme di clausole $S$ è detto saturato rispetto al sistema di inferenze $\phi$ se, 
    per ogni inferenza in $\phi$ con le premesse in $S$, la conclusione dell'inferenza appartiene sempre
    a $S$. 
\end{definition}
Grazie alla proprietà \ref{sup}, si ottiene questa importante proprietà:
\begin{property}
    Un insieme di clausole è \textbf{insoddisfacibile} se, e solo se, il più piccolo insieme di clausole contenente $S$, saturato 
    rispetto al superposition inference system, contiene anche la clausola vuota $\square$
\end{property}

Per rendere efficiente l'algoritmo di saturazione, oltre a queste regole, vengono introdotte
regole di semplificazione per eliminare ridondanze e semplificare le inferenze. Alcune di queste regole sono: \emph{demodulation},
\emph{branch demodulation} e \emph{subsumption resolution}. 

In generale, un algoritmo di saturazione è composto dalle seguenti fasi:
\begin{enumerate}
    \item Viene inizializzato un insieme $S$ in cui vengono memorizzate le clausole
    \item Viene selezionata un'inferenza che può essere applicata a delle clausole presenti in $S$
    \item Nel caso fosse generato un risultato, allora viene aggiunto a $S$
    \item Se viene trovata una clausola vuota $\square$, allora il problema è insoddisfacibile
\end{enumerate}
Se un'inferenza genera una clausola, allora è detta generatrice. Le inferenze generatrici 
sono strettamente collegate alle regole di semplificazioni.
\textsc{Vampire} implementa tre tipi di algoritmi di saturazione: \emph{limited resource strategy}(lrs), \emph{otter} e \emph{discount}.
Tutti fanno parte della famiglia dei \emph{given clause algorithm}.
\begin{algorithm}
    \caption{Given clause algorithm}
    \begin{algorithmic}[1]
        \State \textbf{var} active,passive : sets of clause 
        \State \textbf{var} current,new : clause
        \State active = $\emptyset$
        \State passive = set of input clauses
        \While{passive $\neq \emptyset$}
            \State current = \emph{select}(passive)
            \State passive = passive $\setminus$ $\{\text{current}\}$
            \State active = active $\cup$ $\{\text{current}\}$
            \State new = \emph{infer}(current,active)
            \If{new is $\square$}
                \State \underline{\textbf{return} provable}
            \EndIf
            \State passive = passive $\cup$ $\{\text{new}\}$ 
        \EndWhile
        \State \underline{\textbf{return} unprovable}
    \end{algorithmic}
\end{algorithm}
\cite{kovacs2013first,riazanov2002design,reger2016new}