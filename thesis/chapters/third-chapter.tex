\chapter{Implementazione della procedura}
Come descritto nel precedente capitolo, la procedura di decisione per problemi
appartenenti al frammento \emph{guarded} è formata da una fase di \emph{preprocessing} e 
una fase di risoluzione con $\prec$-\emph{ordered resolution} e \emph{factorization}.

Per la fase di preprocessing è stato costruito un \emph{preprocessor} per trattare problemi 
appartenenti al frammento \emph{guarded}. Questa componente è stata implementata seguendo l'architettura 
del \emph{preprocessor} predefinito.
\begin{algorithm}
    \caption{Preprocessing}
    \begin{algorithmic}
        \State \textbf{var} problem : sets of guarded formula
        \State \textbf{var} result : sets of guarded clause 
        \State problem = \emph{NNF}(problem)
        \State problem = \emph{flatten}(problem)
        \State problem =$\text{\emph{Struct}}_\forall$(problem)
        \State problem = \emph{skolemize}(problem)
        \State result = \emph{clausify}(problem)
        \State \underline{\textbf{return} result}
    \end{algorithmic}
\end{algorithm}

\emph{NNF, flatten, skolemize} e \emph{clausify} (descritte in \ref{sec-prepro}) sono le funzioni di 
\textsc{Vampire} che vengono sfruttate per la trasformazione del problema.
Invece la funzione $\text{\emph{Struct}}_\forall$ viene implementata ex-novo in quanto non è presente 
nel sistema. La funzione è ricorsiva per sfruttare la struttura ad albero delle formule.

\begin{algorithm}
    \caption{$\text{\emph{Struct}}_\forall$}
    \begin{algorithmic}
        \State \textbf{var} input, new, $\alpha$ : formula
        \State \textbf{var} $\bar{y}$ : sets of guarded clause 
        \State $\bar{y} =$ \emph{saveFreeVariable}(input)
        \State $\alpha =$ \emph{createNewLiteral}($\bar{y}$)
        \State new = \emph{generateNewFormula}($alpha$,input)
        \State \emph{addFormulaToUnits}(new)
        \State input = $\alpha$
        \State \underline{\textbf{return} input}
    \end{algorithmic}
\end{algorithm}
Si scorre tutto l'albero e nel caso venisse trovata una formula del tipo $\forall\bar{x}(a \lor A)$ si procede come 
descritto nella definizione \ref{struct-def}:
\begin{enumerate}
    \item si sostituisce la formula con un nuovo predicato $\alpha(\bar{y})$
    \item si aggiunge la formula $\forall \bar{x}\bar{y}(\lnot a \lor \alpha \lor A)$
\end{enumerate}


