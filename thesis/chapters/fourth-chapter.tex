\chapter{Risultati sperimentali}
In questo capitolo vengono presentati i risultati sperimentali del confronto 
tra \textsc{Vampire} originale e \textsc{Vampire} con la nuova procedura di decisione.

Per gli esperimenti sono stati utilizzati i problemi della libreria TPTP versione 8.2.0 \cite{Sut17}.
I problemi presi in esame sono privi di uguaglianza, poiché la procedura di decisione tratta 
solo questo tipo di problemi. 
\'E possibile dividere i problemi in due insiemi:
uno comprende quelli composti da formule generiche di logica del primo ordine (FOF), mentre l'altro 
comprende i problemi composti da clausole (CNF).
\begin{table}[H]
    \begin{center}
    \caption{Numero di problemi}
    \begin{tabular}{ccc}
        \toprule % <-- Toprule here
        &\textbf{FOF} & \textbf{CNF}\\
        \midrule % <-- Midrule here
        Senza equality & 1969 & 2274\\
        \bottomrule % <-- Bottomrule here
        \textbf{\emph{Guarded}} & \textbf{52} & \textbf{62}\\
    \end{tabular}
    \label{tab:prob}
\end{center}
\end{table} 
Entrambi gli insiemi sono stati dati in input 
al classificatore (capitolo \ref{class}) e il risultato è quello riportato in tabella \ref{tab:prob}:
52 problemi di tipo FOF su 1969 e 62 problemi di tipo CNF su 2274 sono \emph{guarded}.
7 di questi 52 problemi di tipo FOF non vengono presi in considerazione nei grafici seguenti, in quanto 
non possono essere risolti in un tempo limite stabilito. Di seguito viene fornita una numerazione 
dei problemi FOF e CNF in modo da rendere più chiara la lettura dei grafici.
\noindent\begin{table}[H]
    \caption{Numerazione dei problemi FOF}
    \resizebox{\linewidth}{!}{
        \begin{tabular}{cccccc}
            \csvautotabular{data/problem.csv} &
            \csvautotabular{data/problem2.csv} &
            \csvautotabular{data/problem3.csv}
        \end{tabular}}
        
\end{table}
\begin{table}[H]
    \caption{Numerazione dei problemi CNF}
    \resizebox{\linewidth}{!}{
        \begin{tabular}{cccccccc}
            \csvautotabular{data/problem-cnf.csv} &
            \csvautotabular{data/problem-cnf2.csv} &
            \csvautotabular{data/problem-cnf3.csv} & 
            \csvautotabular{data/problem-cnf4.csv} 
        \end{tabular}}
        
\end{table}
Nei test, \textsc{Vampire} con la nuova procedura di decisione è indicato con \textsc{Guarded}.
I primi test sono stati eseguiti con le seguenti opzioni di \textsc{Vampire}:
\begin{center}
    \verb|--sa otter -t 10m -m 8000|    
\end{center}
L'algoritmo di saturazione scelto è \emph{otter}, il tempo massimo di timeout è 10 minuti 
e il limite massimo di memoria occupabile è 8000 Mb.
\section{Problemi FOF}
\begin{figure}[H]
    \centering
    \begin{tikzpicture}
        \begin{axis}[
            width=\textwidth,
            xlabel={Problema},
            ylabel={Tempo (s)},
            legend style={at={(0.5,1)},       
            anchor=north,legend columns=-1},
            xtick=data,
            axis x line=bottom,     
            axis y line=left,
            xmajorgrids=true,     
            grid style=dashed,
            xtick align=outside,
            x tick label style = {
                %yshift={-mod(\ticknum,2)*0.1cm}, 
                font = \small, rotate=70, align = center},
        ]
        \legend{\textsc{Vampire}, \textsc{Guarded}}
        \addplot[color=red]table{data/1-vampire-time.dat};
        \addplot[color=blue]table{data/1-guarded-time.dat};
        \end{axis}
    \end{tikzpicture}
    \caption{Tempo impiegato da \textsc{Vampire} e \textsc{Guarded} per risolvere problemi FOF}
\end{figure}
Dai primi test sui problemi FOF emerge che, nella maggior parte dei problemi, \textsc{Guarded} è efficiente
tanto quanto \textsc{Vampire} in termini di tempo. La differenza tra i due, essendo nell'ordine dei millisecondi,
non è significativa. 
Ci sono alcune eccezioni: i problemi MED011+1 (n.5) e NLP263+1 (n.6) sono a favore di
\textsc{Vampire} rispettivamente di circa 10 e 18 secondi, mentre il problema SYO525+1.018 (n.45) è a favore di \textsc{Guarded} di circa 12 secondi.

\begin{figure}[H]
    \begin{tikzpicture}
        \begin{axis}[
            width=\textwidth,
            xlabel={Problema},
            ylabel={Memoria (Kb)},
            legend style={at={(0.5,1)},       
            anchor=north,legend columns=-1},
            every y tick scale label/.style={
                at={(0.07,1)},anchor=north east,inner sep=0pt},
            xtick=data,
            axis x line=bottom,     
            axis y line=left,
            xmajorgrids=true,     
            grid style=dashed,
            xtick align=outside,
            x tick label style = {
                %yshift={-mod(\ticknum,2)*0.1cm}, 
                font = \small, rotate=70, align = center},
        ]
        \legend{\textsc{Vampire}, \textsc{Guarded}}
        \addplot[color=red]table{data/1-vampire-mem.dat};
        \addplot[color=blue]table{data/1-guarded-mem.dat};
        \end{axis}
    \end{tikzpicture}
    \caption{Memoria occupata da \textsc{Vampire} e \textsc{Guarded} per risolvere problemi FOF}
\end{figure}

\textsc{Vampire} e \textsc{Guarded} sono molto simili anche in termini di memoria occupata per la risoluzione di un problema.
L' unica eccezione rilevante 
è il problema NLP263+1 (n.6) in cui \textsc{Guarded} occupa $6.7\times10^5$ Kb di memoria, mentre \textsc{Vampire} solo $5.2\times10^5$ Kb.\\\\
Alla fine di questa batteria di test, l'analisi ha posto particolare attenzione alle eccezioni. 
\begin{description}
    \item[Memoria:] La tendenza di \textsc{Guarded} ad occupare più memoria in alcuni problemi proseguirà anche nei prossimi test
    poiché, in problemi complessi e di grandi dimensioni, la trasformazione strutturale 
    $\text{\emph{Struct}}_\forall$ aggiunge nuove formule del tipo $\forall\bar{x}\bar{y}(\lnot a \lor \lnot\alpha \lor A)$ 
    quindi occupa molto più spazio rispetto a \textsc{Vampire}.
    \item[Tempo:] Nelle eccezioni in cui \textsc{Vampire} impiega meno tempo, è utilizzata
    una tecnica di \emph{preprocessing} detta \emph{unused predicate definition removal}.
    Una definizione di un predicato è una formula del tipo 
    \[\forall(X_1,\dots,X_n) : p(X_1, \dots, X_n) \leftrightarrow F\]
    in cui $p$ predicato non risulta in $F$. 
    Con questa tecnica \textsc{Vampire} si può comportare in 3 modi:
    \begin{enumerate}
        \item se $p$ è presente nel resto del problema solo in forma positiva, allora $\leftrightarrow$ della definizione
        viene sostituito da $\rightarrow$;
        \item se $p$ non è presente nel resto del problema, allora la definizione può essere eliminata;
        \item altrimenti la definizione non viene modificata
    \end{enumerate}
    In tal modo \textsc{Vampire} riesce a risolvere i problemi FOF in modo più rapido.
\end{description}

Nei prossimi test, la tecnica di \emph{unused predicate definition removal} (updr) 
verrà disattivata da \textsc{Vampire} infatti le opzioni saranno:
\begin{center}
    \verb|-updr off --sa otter -t 10m -m 8000|    
\end{center}

\begin{figure}[H]
    \begin{tikzpicture}
        \begin{axis}[
            width=\textwidth,
            xlabel={Problema},
            title style={align=center},
            ylabel={Tempo (s)},
            legend style={at={(0.5,1)},       
            anchor=north,legend columns=-1},
            xtick=data,
            axis x line=bottom,     
            axis y line=left,
            xmajorgrids=true,     
            grid style=dashed,
            xtick align=outside,
            x tick label style = {
                %yshift={-mod(\ticknum,2)*0.1cm}, 
                font = \small, rotate=70, align = center},
        ]
        \legend{\textsc{Vampire}, \textsc{Guarded}}
        \addplot[color=red]table{data/2-vampire-time.dat};
        \addplot[color=blue]table{data/2-guarded-time.dat};
        \end{axis}
    \end{tikzpicture}
    \caption{Tempo impiegato da \textsc{Vampire} e \textsc{Guarded} per risolvere problemi FOF con \emph{updr} disattivata}
\end{figure}

Con la disattivazione della \emph{updr}, le prestazioni di \textsc{Vampire} peggiorano significativamente. 
Infatti, in questo modo \textsc{Vampire} impiega più tempo di \textsc{Guarded}, con una 
differenza di circa 10 secondi nel problema MED011+1 (n.5) e di circa 20 secondi in NLP263+1 (n.6). In questi test \textsc{Vampire}
impiega circa 20 secondi in più nel problema MED011+1 (n.5) e circa 38 secondi in più nel problema NLP263+1 (n.6) rispetto alla batteria di 
test precedente.\\\\
\begin{figure}[H]
    \begin{tikzpicture}
        \begin{axis}[
            width=\textwidth,
            title style={align=center},
            xlabel={Problema},
            ylabel={Memoria (Kb)},
            legend style={at={(0.5,1)},       
            anchor=north,legend columns=-1},
            every y tick scale label/.style={
                at={(0.07,1)},anchor=north east,inner sep=0pt},
            xtick=data,
            axis x line=bottom,     
            axis y line=left,
            xmajorgrids=true,     
            grid style=dashed,
            xtick align=outside,
            x tick label style = {
                %yshift={-mod(\ticknum,2)*0.1cm}, 
                font = \small, rotate=70, align = center},
        ]
        \legend{\textsc{Vampire}, \textsc{Guarded}}
        \addplot[color=red]table{data/2-vampire-mem.dat};
        \addplot[color=blue]table{data/2-guarded-mem.dat};
        \end{axis}
    \end{tikzpicture}
    \caption{Memoria occupata da \textsc{Vampire} e \textsc{Guarded} per risolvere problemi FOF con \emph{updr} disattivata}
\end{figure}


Disattivando l'\emph{updr}, la quantità di memoria occupata da \textsc{Vampire} per il problema NLP263+1 aumenta al punto 
da superare quella utilizzata da \textsc{Guarded}. Nonostante ciò, la differenza non è da considerare significativa.

\section{Problemi CNF}
\begin{figure}[H]
    \begin{tikzpicture}
        \begin{axis}[
            width=\textwidth,
            xlabel={Problema},
            ylabel={Tempo (s)},
            legend style={at={(0.7,1)},       
            anchor=north,legend columns=-1},
            xtick=data,
            axis x line=bottom,     
            axis y line=left,
            xmajorgrids=true,     
            grid style=dashed,
            xtick align=outside,
            x tick label style = {
                yshift={-mod(\ticknum,2)*0.35cm}, 
                font = \small, rotate=90, align = center},
        ]
        \legend{\textsc{Vampire}, \textsc{Guarded}}
        \addplot[color=red]table{data/1-cnf-vampire-time.dat};
        \addplot[color=blue]table{data/1-cnf-guarded-time.dat};
        \end{axis}
    \end{tikzpicture}
    \caption{Tempo impiegato da \textsc{Vampire} e \textsc{Guarded} per risolvere problemi CNF}
\end{figure}


Dai test sui problemi CNF emergono gli stessi risultati dei precedenti test sui problemi FOF: nella maggior parte dei problemi, \textsc{Guarded} è efficiente
quanto \textsc{Vampire}, sia in termini di tempo che di memoria. In questo caso, la differenza è ancora minore rispetto ai problemi FOF.
L'unica eccezione è il problema PUZ036-1.005 (n.25) in cui \textsc{Vampire} impiega circa 0,2 secondi e occupa 1100 Kb, mentre \textsc{Guarded}
circa 4 secondi e 15000 Kb.

\begin{figure}[H]
    \begin{tikzpicture}
        \begin{axis}[
            width=\textwidth,
            xlabel={Problema},
            ylabel={Memoria (Kb)},
            legend style={at={(0.7,1)},       
            anchor=north,legend columns=-1},
            every y tick scale label/.style={
                at={(0.07,1)},anchor=north east,inner sep=0pt},
            xtick=data,
            axis x line=bottom,     
            axis y line=left,
            xmajorgrids=true,     
            grid style=dashed,
            xtick align=outside,
            x tick label style = {
                yshift={-mod(\ticknum,2)*0.35cm}, 
                font = \small, rotate=70, align = center},
        ]
        \legend{\textsc{Vampire}, \textsc{Guarded}}
        \addplot[color=red]table{data/1-cnf-vampire-mem.dat};
        \addplot[color=blue]table{data/1-cnf-guarded-mem.dat};
        \end{axis}
    \end{tikzpicture}
    \caption{Memoria occupata da \textsc{Vampire} e \textsc{Guarded} per risolvere problemi CNF}
\end{figure}

Alla fine di questi primi test sui problemi CNF, è stata effettuata l'analisi sul problema PUZ036-1.005 (n.25).
\textsc{Vampire}, nell'algoritmo di saturazione, utilizza la \emph{forward subsumption},
un'inferenza che fa parte delle \emph{forward simplification} (capitolo \ref{sec-kernel}) utile per eliminare le ridondanze.
Questa tecnica permette a \textsc{Vampire} di essere più veloce ed efficiente.

Nei prossima batteria di test, la tecnica di \emph{forward subsumption} (fs) verrà disattivata da \textsc{Vampire} 
infatti le opzioni saranno:
\begin{center}
    \verb|-fs off --sa otter -t 10m -m 8000|    
\end{center}
Si noti che i test sono stati effettuati anche con \emph{updr} disattivata, ma
non stati riportati i risultati poiché non c'era nessun cambiamento significativo. Lo stesso 
vale anche per l'opzione \emph{fs} nei problemi FOF. Infatti, su questi problemi, sono stati 
eseguiti i test anche con \emph{fs} disattivata ma i risultati non hanno portato a nessun cambiamento rilevante.

\begin{figure}[H]
    \begin{tikzpicture}
        \begin{axis}[
            width=\textwidth,
            xlabel={Problema},
            title style={align=center},
            ylabel={Tempo (s)},
            legend style={at={(0.7,1)},       
            anchor=north,legend columns=-1},
            xtick=data,
            axis x line=bottom,     
            axis y line=left,
            xmajorgrids=true,     
            grid style=dashed,
            xtick align=outside,
            x tick label style = {
                yshift={-mod(\ticknum,2)*0.35cm}, 
                font = \small, rotate=90, align = center},
        ]
        \legend{\textsc{Vampire}, \textsc{Guarded}}
        \addplot[color=red]table{data/3-cnf-vampire-time.dat};
        \addplot[color=blue]table{data/3-cnf-guarded-time.dat};
        \end{axis}
    \end{tikzpicture}
    \caption{Tempo impiegato da \textsc{Vampire} e \textsc{Guarded} per risolvere problemi CNF con \emph{forward subsumption} disattivata per entrambi}
\end{figure}

Con la disattivazione della \emph{forward subsumption}, \textsc{Vampire} peggiora sia in termini 
di tempo che di memoria occupata nel problema PUZ036-1.005 (n.25). La differenza di tempo tra \textsc{Vampire}
e \textsc{Guarded} si assottiglia molto in questo problema, infatti è di circa mezzo secondo.

\begin{figure}[H]
    \begin{tikzpicture}
        \begin{axis}[
            width=\textwidth,
            xlabel={Problema},
            title style={align=center},
            ylabel={Memoria (Kb)},
            legend style={at={(0.7,1)},       
            anchor=north,legend columns=-1},
            every y tick scale label/.style={
                at={(0.07,1)},anchor=north east,inner sep=0pt},
            xtick=data,
            axis x line=bottom,     
            axis y line=left,
            xmajorgrids=true,     
            grid style=dashed,
            xtick align=outside,
            x tick label style = {
                yshift={-mod(\ticknum,2)*0.35cm}, 
                font = \small, rotate=90, align = center},
        ]
        \legend{\textsc{Vampire}, \textsc{Guarded}}
        \addplot[color=red]table{data/3-cnf-vampire-mem.dat};
        \addplot[color=blue]table{data/3-cnf-guarded-mem.dat};
        \end{axis}
    \end{tikzpicture}
    \caption{Memoria occupata da \textsc{Vampire} e \textsc{Guarded} \\per risolvere problemi CNF con \emph{forward subsumption} disattivata per entrambi}
\end{figure}

Con la disattivazione della \emph{forward subsumption}, \textsc{Vampire} e \textsc{Guarded} occupano  
circa la stessa quantità di memoria.

