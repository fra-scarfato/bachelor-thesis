\chapter{Conclusioni}\label{fourth-c}
In conclusione, quello che è emerso dal capitolo \ref{third-c} è che 
\textsc{Vampire} originale e \textsc{Vampire} esteso con la nuova procedura di decisione sono indistinguibili,
con un'attenzione però alle potenzialità della procedura nei problemi più complessi. 
\'E importante sottolineare che il benchmark effettuato 
è poco significativo, in quanto sono stati selezionati problemi senza uguaglianza, di cui solo
pochi fanno parte del frammento \emph{guarded}.\\\\
Sarebbe possibile approfondire l'analisi della procedura di decisione tramite la generazione di problemi complessi che fanno 
parte della logica modale. 

Inoltre, grazie al classificatore (cap \ref{class}), si è osservato che circa 640 problemi in cui è presente l'uguaglianza sono \emph{guarded}. 
Quindi, un'ulteriore indagine può essere condotta sullo studio di \citeauthor{ganzinger1999superposition}, con l'implementazione della 
procedura di decisione basata sulla \emph{superposition} \cite{ganzinger1999superposition}. In questo modo, sarebbe possibile 
includere nell'analisi quei problemi \emph{guarded} in cui è presente l'uguaglianza. 