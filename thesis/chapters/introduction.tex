\pagestyle{introduction}
\chapter*{Introduzione}
\addcontentsline{toc}{chapter}{Introduzione}
Un \emph{theorem prover} è uno strumento molto importante che permette di verificare formalmente 
sistemi software e hardware ed è strettamente collegato ai campi della logica e della matematica.
I \emph{theorem prover} sono degli strumenti \emph{general purpose} che puntano a risolvere il maggior numero di problemi nel modo più veloce 
ed efficiente possibile. Per migliorare questi tool, la ricerca si sta muovendo verso la definizione di 
decisione di procedure per problemi decidibili. Una decisione di procedura è un algoritmo che verifica 
la soddisfacibilità di un problema e termina sempre. Un sottoinsieme di problemi decidibili nella logica del primo ordine è detto 
frammento. La ricerca di queste procedure di decisione è essenziale poiché, in alcuni casi, questi approcci possono risultare 
più efficienti di una strategia general purpose. L'ideale sarebbe inglobare, quando è possibile, queste procedure di decisione nei 
theorem prover e usarle quando un problema appartiene a un determinato frammento.

In questa tesi viene presentata un'implementazione di una procedura di decisione per il frammento \emph{guarded} su   
\textsc{Vampire}, uno degli \emph{automated theorem prover} più famoso ed efficiente in circolazione per la risoluzione di 
problemi di logica del primo ordine. Il documento è diviso in quattro capitoli: 
\begin{enumerate}
    \item Il primo capitolo fornisce un'introduzione ai concetti base di \textsc{Vampire} propedeutici ai capitoli successivi
    \item Il secondo capitolo formalizza tutti i concetti riguardanti il frammento \emph{guarded}
    \item Il terzo capitolo descrive l'implementazione della procedura di decisione in \textsc{Vampire}
    \item Il quarto capitolo presenta i risultati ottenuti dal confronto tra \textsc{Vampire} e \textsc{Vampire} con la nuova procedura di decisione
\end{enumerate}

