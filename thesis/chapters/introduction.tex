\pagestyle{introduction}
\chapter*{Introduzione}
\addcontentsline{toc}{chapter}{Introduzione}
Un \emph{theorem prover} è un importante strumento che permette di verificare formalmente 
sistemi software e hardware, strettamente collegato ai campi della logica e della matematica.
I \emph{theorem prover} sono degli strumenti \emph{general purpose} che puntano a risolvere il maggior numero di problemi nel modo più veloce 
ed efficiente possibile. Per migliorare questi sistemi, la ricerca si sta muovendo verso la definizione di 
procedure di decisione per problemi decidibili. Una procedura di decisione è un algoritmo che verifica 
la soddisfacibilità di un problema e arriva sempre a una terminazione. Nella logica del primo ordine, un sottoinsieme di problemi decidibili è detto 
frammento. La ricerca di queste procedure di decisione è rilevante poiché, in alcuni casi, questi approcci possono risultare 
più efficienti di una strategia general purpose. L'ideale sarebbe inglobare queste procedure di decisione nei 
theorem prover, quando è possibile, e utilizzarle nel caso in cui un problema appartenga a un determinato frammento.

In questa tesi viene presentata un'implementazione di una procedura di decisione per il frammento \emph{guarded} su   
\textsc{Vampire}, uno degli \emph{automated theorem prover}, per la risoluzione di 
problemi di logica del primo ordine, più conosciuti ed efficienti in circolazione. Nella prima parte vengono introdotte le caratteristiche base
più importanti di \textsc{Vampire} e del frammento \emph{guarded}. Nella seconda parte viene descritta 
l'implementazione sperimentale della procedura di decisione per questo particolare frammento. Infine, viene presentata un'analisi basata sul 
confronto tra il software originario e la versione estesa con la procedura di decisione. 
