\let\cleardoublepage\relax
\chapter{Frammento Guarded}
Il frammento \emph{guarded} è un frammento della logica del primo ordine introdotto da Hajnal Andréka,
István Nèmeti e Johan van Benthem. L'importanza è dovuta alla possibilità  di tradurre 
molte logiche modali, con significative proprietà computazionali e logiche, in formule del primo ordine 
appartenenti al frammento \emph{guarded}. \'E stato provato che questo frammento (con uguaglianza e non) 
è decidibile in \textsc{2-EXPTIME}. 

Sono state proposte numerose generalizzazioni di questo frammento in quanto
alcune logiche modali importanti (come quella temporale) non possono essere tradotte. La generalizzazione 
piu datata è il frammento \emph{loosely guarded} che ha dei vincoli meno stringenti sulla definizione di guardia.
L'implementazione della procedura di decisione in questa tesi riguarda il frammento \emph{guarded} senza uguaglianza \cite{de2003deciding}.
\begin{definition}
    Il frammento \emph{guarded} è ricorsivamente definito come i seguenti sottoinsiemi di logica del primo ordine senza
    uguaglianza e simboli di funzione:
    \begin{enumerate}
        \item $\top,\bot\in$ GF.
        \item Se $a$ è una formula atomica, allora $a\in$ GF.
        \item GF è chiuso rispetto a $\lnot, \land,
        \lor, \rightarrow, \leftrightarrow$.
        \item Sia $A\in$ GF, $a$ formula atomica tale che ogni variabile libera di $A$ occorra almeno una volta negli argomenti 
        di $a$. Allora 
        \begin{itemize}
            \item $\forall\bar{x}(a \rightarrow A)\in$ GF,
            \item $\exists\bar{x}(a \land A)\in$ GF,
            \item $\forall\bar{x}(a \lor A)\in$ GF (negazione della precedente)
        \end{itemize}
        con la formula atomica $a$ definita \textbf{guardia}.
    \end{enumerate}
\end{definition}
Dopo aver delineato quando una formula del primo ordine appartiene al frammento \emph{guarded}, di seguito viene 
definito quando una clausola è \emph{guarded}.
\begin{definition}
    Una clausola $c$ è \emph{guarded} se soddisfa le seguenti condizioni:
    \begin{enumerate}
        \item Ogni termine funzionale non \emph{ground} in $c$ contiene tutte le variabili di $c$.
        \item Se $c$ è non ground allora c'è un letterale negativo $\lnot A$ in $c$ che contiene tutte le variabili di $c$ ma 
        non contiene termini funzionali non \emph{ground}.

        Il letterale negativo $\lnot A$ è definito \textbf{guardia}.
    \end{enumerate}
    Se la clausola $c$ è \emph{ground}, allora è \emph{guarded}.
\end{definition}
\begin{definition}
    Una insieme formato da clausole \emph{guarded} è anch'esso \emph{guarded}
\end{definition}
\section{Preprocessing}
Come menzionato nel precedente capitolo \ref{sec-prepro}, il problema viene pre-processato per ottenere un insieme di clausole 
da sottomettere all'algoritmo di risoluzione. Le operazioni di \emph{preprocessing} sono le seguenti:
\begin{enumerate}
    \item \emph{Negation normal form} 
    \item $\text{\emph{Struct}}_\forall$
    \item \emph{Skolemization}
    \item \emph{Clausification}
\end{enumerate} 
L'unica operazione non ancora definita è $\text{\emph{Struct}}_\forall$.
\begin{definition}\label{struct-def}
    $\text{\emph{Struct}}_\forall$ è una trasformazione strutturale che è ottenuta da:
    \begin{enumerate}
        \item la sostituzione delle sotto-formule $\forall\bar{x}(a \rightarrow A)$, $\forall\bar{x}(a \lor A)$ che hanno 
        $\bar{y}$ variabili libere, con un nuovo nome $\alpha(\bar{y})$
        \item l'aggiunta della formula $\forall\bar{x}\bar{y}(\lnot a \lor \lnot\alpha \lor A)$
    \end{enumerate}
\end{definition}
Queste operazioni, oltre a generare un insieme di clausole, preservano l'appartenenza delle formule al frammento \emph{guarded}.
\begin{theorem}
    Sia $F\in$ GF allora 
    \begin{enumerate}
        \item $F'=\text{NNF}(F) \in$ GF,
        \item $F''=\text{\emph{Struct}}_\forall(F') \in$ GF,
        \item applicando skolemization e clausification a $F''$ si ottiene un insieme di clausole guarded.
    \end{enumerate}
\end{theorem}
Il teorema è dimostrato nello studio di \citeauthor{de2003deciding} in \cite{de2003deciding}.
\section{Resolution}
La procedure di decisione per il frammento \emph{guarded} necessita della definizione di una nuova regola di risoluzione.
Prima di definire tale regola, è necessario stabilire: un ordinamento e dei parametri secondo cui ordinare.
\begin{definition}
    La \emph{Vardepth} di un termine/atomo $A$ è definita ricorsivamente come segue:
    \[\text{\emph{Vardepth}}(A) = 
    \begin{cases}
        -1 & \text{se $A$ è ground}\\
        0 & \text{se $A$ è una variabile}\\
        \max\{1+\text{\emph{Vardepth}}(t_1), \dots, 1+\text{\emph{Vardepth}}(t_n)\} & \text{se }A=f(t_1,\dots,t_n)
    \end{cases}\]
\end{definition}
\begin{definition}
    La \emph{Vardepth} di un letterale equivale alla \emph{Vardepth} di un atomo.
\end{definition}
\begin{definition}
    La \emph{Vardepth} di una clausola $c$ equivale alla \emph{Vardepth} massimale di un letterale in $c$. 
\end{definition}
\begin{definition}
    Sia $A$ atomo, letterale o clausola, allora \emph{Var}$(A)$ è definito come l'insieme delle 
    variabili che occorrono in $A$.
\end{definition}
Una volta definiti questi due parametri \emph{Vardepth} e \emph{Var}, è possibile definire l'ordinamento.
\begin{definition}
    Definiamo l'ordinamento $\prec$ sui letterali $A,B$ come segue:
    \[A\prec B \quad\text{se}\quad(\text{\emph{Vardepth}}(A)<\text{\emph{Vardepth}}(B)) \lor (\text{\emph{Var}}(A)\subset\text{\emph{Var}}(B))\]
\end{definition}
Ora è possibile introdurre la nuova regola di risoluzione definita \emph{ordered resolution rule}.
\begin{definition}
    Sia $\prec$ ordinamento sui letterali, $\{A_1\}\cup R_1$ e $\{\lnot A_2\}\cup R_2$ clausole allora se:
    \begin{enumerate}
        \item $\{A_1\}\cup R_1$ e $\{\lnot A_2\}\cup R_2$ non hanno variabili in comune,
        \item $A_1$ deve essere massimale in $R_1$ ovvero $\nexists A\in R_1 : (A_1 \prec A)$,
        \item $A_2$ deve essere massimale in $R_2$ ovvero $\nexists A\in R_2 : (A_2 \prec A)$,
        \item $A_1$ e $A_2$ hanno un unificatore più generale (\emph{mgu}) $\theta$,
    \end{enumerate}
    allora $R_1\theta \cup R_2\theta$ è chiamato $\prec$-\emph{ordered resolvent} di $\{A_1\}\cup R_1$ e $\{\lnot A_2\}\cup R_2$.
    \begin{equation}
        \begin{gathered}
            \underline{\{\underline{A_1}\} \lor R_1 \quad\{\underline{\lnot A_2}\}\lor R_2}\\
            (R_1 \lor R_2)\theta
        \end{gathered}
    \end{equation}
\end{definition}
Questa regola è una \emph{resolution rule} (\ref{res-eq}) ristretta da un'ordinamento $\prec$. \'E possibile
definire anche una  \emph{factorization rule} (\ref{fact-eq}) ristretta.
\begin{definition}
    Sia $\prec$ ordinamento sui letterali, $\{A_1,A_2\}\cup R$ clausola allora se:
    \begin{enumerate}
        \item $A_1$ deve essere massimale in $R$ ovvero $\nexists A\in R : (A_1 \prec A)$,
        \item $A_1$ e $A_2$ hanno un unificatore più generale (\emph{mgu}) $\theta$,
    \end{enumerate}
    allora $\{A_1\theta\}\cup R\theta$ è chiamato $\prec$-\emph{ordered factor} di $\{A_1,A_2\}\cup R$.
    \begin{equation}
        \begin{gathered}
            \underline{\{\underline{A_1} \lor \underline{A_2}\} \lor R}\\
            (A_1 \lor R)\theta
        \end{gathered}
    \end{equation}
\end{definition}
Nell'algoritmo di risoluzione è necessaria una \emph{resolution rule} ristretta dall'ordinamento $\prec$ poiché preserva 
la proprietà degli $\prec$-\emph{ordered resolvent} di essere \emph{guarded}. Invece la \emph{factorization rule}
\textbf{preserva la proprietà delle clausole di essere \emph{guarded} anche senza applicare una restrizione}.
Tutto questo viene provato dal seguente teorema per la cui dimostrazione si rimanda a \cite{de2003deciding}:
\begin{theorem}
    \begin{enumerate}
        \item Se $c_1$ e $c_2$ sono clausole \emph{guarded} allora $c$ $\prec$-\emph{ordered resolvent} di $c_1$ e $c_2$ è
        \emph{guarded}
        \item Se $c_1$ è una clausola \emph{guarded} allora $c$ \emph{factor} di $c_1$ è \emph{guarded}
    \end{enumerate}
\end{theorem}
Un algoritmo di risoluzione, per essere definito procedura di decisione, deve terminare ed essere completo.

L'algoritmo con la $\prec$-\emph{ordered resolution rule} e la \emph{factorization rule} termina perché, grazie alla restrizione
sulla \emph{resolution rule}, è possibile derivare solo un insieme finito di clausole da un insieme finito di clausole \emph{guarded}. 
\begin{lemma}
    Sia $C$ un insieme finito di clausole \emph{guarded}. Se $\bar{C}$ è l'insieme generato da $C$ tramite 
    la $\prec$-\emph{ordered resolution rule} e la \emph{factorization rule}, allora $\bar{C}$ ha dimensione finita.
\end{lemma}
La dimostrazione di questo lemma è possibile poiché il numero di variabili e la \emph{Vardepth} sono limitati superiormente \cite{de2003deciding}. 

\begin{definition}
    Un algoritmo di risoluzione è completo per una logica se riesce a risolvere ogni problema appartenente
    a quella logica. 
\end{definition}
Nello studio di \citeauthor{de2003deciding} è presentata la dimostrazione della completezza per la procedura di decisione \cite{de2003deciding}.